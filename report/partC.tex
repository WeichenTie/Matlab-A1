\section{Part C: Robot Speed Limits}
\subsection{Approach to Calculation in Matlab}
A method of relating the joint positions $q_1,q_2,q_3,q_4,q_5,q_6$ and joint velocities $\dot{q_1},\dot{q_2},\dot{q_3},\dot{q_4},\dot{q_5},\dot{q_6}$ to the end effector's linear and angular velocities, $\dot{x},\dot{y},\dot{z},\dot{\theta_1},\dot{\theta_2},\dot{\theta_3}$ respectively is through the Jacobian matrix $\textbf{J}$ where,
\begin{equation*}
    \begin{pmatrix}\dot{x}\\\dot{y}\\\dot{z}\\\dot{\theta_1}\\\dot{\theta_2}\\\dot{\theta_3}\end{pmatrix} = \textbf{J}_{q_1, q_2, q_3, q_4, q_5, q_6} \begin{pmatrix}\dot{q_1}\\\dot{q_2}\\\dot{q_3}\\\dot{q_4}\\\dot{q_5}\\\dot{q_6}\end{pmatrix}
\end{equation*}
\\ We first define the following DH Table for the UR5e robot.
\begin{table}[H]
    \centering
    \begin{tabular}{|c|c|c|c|c|}
        \hline
                & \textbf{theta (rad)} & \textbf{a (m)} & \textbf{d (m)} & \textbf{alpha (rad)} \\ \hline
        Joint 1 & $q_1$                & 0              & 0.1625         & $\pi$/2              \\ \hline
        Joint 2 & $q_2$                & -0.425         & 0              & 0                    \\ \hline
        Joint 3 & $q_3$                & -0.3922        & 0              & 0                    \\ \hline
        Joint 4 & $q_4$                & 0              & 0.1333         & $\pi$/2              \\ \hline
        Joint 5 & $q_5$                & 0              & 0.0997         & -$\pi$/2             \\ \hline
        Joint 6 & $q_6$                & 0              & 0.0996         & 0                    \\ \hline
    \end{tabular}
    \caption{The DH table for the UR5e robot arm}
    \label{table:DH-UR5e part C}
\end{table}
We can define the parameters of each link in the link array using 'Links' in Matlab's RVC toolbox and then construct the articulated robot using 'SerialLink' the link array.
\\ \\ With the robot constructed, we will now iterate through each set of joint positions and velocities and obtaining the Jacobian matrix using the 'jacob0' function on the robot at the specified joint positions. By multiplying this Jacobian matrix with the corresponding joint velocities, it is possible to derive the linear velocities $\dot{x}$, $\dot{y}$ and $\dot{z}$ for the end effector.
\\ \\ It is then possible to determine the magnitude of this velocity simply through the equation,
\begin{equation*}
    |v| = \sqrt{\dot{x^2} + \dot{y^2} + \dot{z}^2}
\end{equation*}
\\ We then update the value maximum velocity accordingly.
\\ \\ The Matlab code is shown below,

\begin{algorithm}[H]
    function maxLinearVelocity = calculateMaxLinearVelocity(jointPositions,jointVelocities)\\
    \hspace{18pt} \% Defining the link array in accordance with the DH table\\
    \hspace{18pt} L(1) = Link([0, 0.1625, 0,  pi/2]); \% Link 1\\
    \hspace{18pt} L(2) = Link([0, 0, -0.425,  0]); \% Link 2\\
    \hspace{18pt} L(3) = Link([0,  0, -0.3922, 0]); \% Link 3\\
    \hspace{18pt} L(4) = Link([0, 0.1333, 0,  pi/2]); \% Link 4\\
    \hspace{18pt} L(5) = Link([0, 0.0997, 0,  -pi/2]); \% Link 5\\
    \hspace{18pt} L(6) = Link([0, 0.0996, 0,  0]); \% Link 6\\
    \hspace{18pt} \% Creating the robot\\
    \hspace{18pt} robot = SerialLink(L, 'name', 'Articulated');\\
    \hspace{18pt} maxLinearVelocity = 0;\\
    \hspace{18pt} \% Iterate through the jointPositions and jointVelocities arrays.\\
    \hspace{18pt}for i = 1:length(jointPositions)\\
    \hspace{18pt}    \hspace{18pt}\% Calculate the Jacobian matrix\\
    \hspace{18pt}    \hspace{18pt}jacobian = jacob0(robot, jointPositions(i,:));\\
    \hspace{18pt}    \hspace{18pt}\% Calculate tool/end effector linear velocity and angular velocities \\
    \hspace{18pt}    \hspace{18pt}toolVelocity = jacobian * jointVelocities(i,:)';\\
    \hspace{18pt}    \hspace{18pt}\% Tool/end effector linear velocities encoded in first 3 rows of\\
    \hspace{18pt}    \hspace{18pt}\% vector\\
    \hspace{18pt}    \hspace{18pt}linearVelocities = toolVelocity(1:3);\\
    \hspace{18pt}    \hspace{18pt}\% Calculate the magnitude of this velocity\\
    \hspace{18pt}    \hspace{18pt}dotProduct = sum(linearVelocities .* linearVelocities);\\
    \hspace{18pt}    \hspace{18pt}magnitude =sqrt(dotProduct);\\
    \hspace{18pt}    \hspace{18pt}\% Update the maxLinearVelocity value\\
    \hspace{18pt}    \hspace{18pt}maxLinearVelocity = max(maxLinearVelocity, magnitude);\\
    \hspace{18pt}end\\
    end\\
\end{algorithm}
\subsection{Jacobian Calculation for the First Location}

The velocity of the end effector at the first position can also be manually calculated similarly where its corresponding joint position can be read directly from the UR5e robot.
\begin{equation*}
    \begin{bmatrix} q_1&q_2&q_3&q_4&q_5&q_6 \end{bmatrix} = \begin{bmatrix}0 & -1.124& 1.803 & -2.250 & -1.571 & 0 \end{bmatrix}
\end{equation*}
\\We first derive the transformation matrices for each joint position ($\hspace{1pt} ^{0}T_{1} ,\hspace{1pt} ^{0}T_{2},\hspace{1pt} ^{0}T_{3},\hspace{1pt} ^{0}T_{4},\hspace{1pt} ^{0}T_{5},\hspace{1pt} ^{0}T_{6}$) using the joint-to-joint transformation matrix,
\begin{equation*}
    ^{n-1}T_{n}  = \begin{bmatrix}
        \cos(q_n) & -\sin(q_n)\cos(\alpha_n) & \sin(q_n)\sin(\alpha_n)  & a_n \cos(q_n) \\
        \sin(q_n) & \cos(q_n)\cos(\alpha_n)  & -\cos(q_n)\sin(\alpha_n) & a_n \sin(q_n) \\
        0         & \sin(\alpha_n)           & \cos(\alpha_n)           & d_n           \\
        0         & 0                        & 0                        & 1
    \end{bmatrix}
\end{equation*}
Therefore we get the following intermediate matrices,
\begin{equation*}
    \begin{split}
        ^{0}T_{1}                                                 & = \begin{bmatrix}
                                                                          1 & 0 & 0  & 0      \\
                                                                          0 & 0 & -1 & 0      \\
                                                                          0 & 1 & 0  & 0.1625 \\
                                                                          0 & 0 & 0  & 1
                                                                      \end{bmatrix}                                             \\
        ^{0}T_{2} = \hspace{1pt} ^{0}T_{1} \hspace{1pt} ^{1}T_{2} & =                     \begin{bmatrix}
                                                                                              1 & 0 & 0  & 0      \\
                                                                                              0 & 0 & -1 & 0      \\
                                                                                              0 & 1 & 0  & 0.1625 \\
                                                                                              0 & 0 & 0  & 1
                                                                                          \end{bmatrix}\begin{bmatrix}
                                                                                                           0.4319  & 0.9019 & 0 & -0.1835 \\
                                                                                                           -0.9019 & 0.4319 & 0 & 0.3833  \\
                                                                                                           0       & 0      & 1 & 0       \\
                                                                                                           0       & 0      & 0 & 1
                                                                                                       \end{bmatrix} \\
                                                                  & =\begin{bmatrix}
                                                                         0.4319  & 0.9019 & 0  & -0.1835 \\
                                                                         0       & 0      & -1 & 0       \\
                                                                         -0.9019 & 0.4319 & 0  & 0.5458  \\
                                                                         0       & 0      & 0  & 1
                                                                     \end{bmatrix}                                  \\
        ^{0}T_{3} = \hspace{1pt} ^{0}T_{2} \hspace{1pt} ^{2}T_{3} & = \begin{bmatrix}
                                                                          0.4319  & 0.9019 & 0  & -0.1835 \\
                                                                          0       & 0      & -1 & 0       \\
                                                                          -0.9019 & 0.4319 & 0  & 0.5458  \\
                                                                          0       & 0      & 0  & 1
                                                                      \end{bmatrix}\begin{bmatrix}
                                                                                       -0.2301 & -0.9732 & 0 & 0.0902  \\
                                                                                       0.9732  & -0.2301 & 0 & -0.3817 \\
                                                                                       0       & 0       & 1 & 0       \\
                                                                                       0       & 0       & 0 & 1
                                                                                   \end{bmatrix}                    \\
                                                                  & =\begin{bmatrix}
                                                                         0.7784 & -0.6278 & 0  & -0.4888 \\
                                                                         0      & 0       & -1 & 0       \\
                                                                         0.6278 & 0.7784  & 0  & 0.2996  \\
                                                                         0      & 0       & 0  & 1
                                                                     \end{bmatrix}                                  \\                                      \\
        ^{0}T_{4} = \hspace{1pt} ^{0}T_{3} \hspace{1pt} ^{3}T_{4} & =\begin{bmatrix}
                                                                         0.7784 & -0.6278 & 0  & -0.4888 \\
                                                                         0      & 0       & -1 & 0       \\
                                                                         0.6278 & 0.7784  & 0  & 0.2996  \\
                                                                         0      & 0       & 0  & 1
                                                                     \end{bmatrix}\begin{bmatrix}
                                                                                      -0.6278 & 0 & -0.7783 & 0      \\
                                                                                      -0.7783 & 0 & 0.6278  & 0      \\
                                                                                      0       & 1 & 0       & 0.1333 \\
                                                                                      0       & 0 & 0       & 1
                                                                                  \end{bmatrix}                      \\
                                                                  & =\begin{bmatrix}
                                                                         0  & 0  & -1 & -0.4888 \\
                                                                         0  & -1 & 0  & -0.1333 \\
                                                                         -1 & 0  & 0  & 0.2996  \\
                                                                         0  & 0  & 0  & 1
                                                                     \end{bmatrix}                                           \\
        ^{0}T_{5} = \hspace{1pt} ^{0}T_{4} \hspace{1pt} ^{4}T_{5} & =\begin{bmatrix}
                                                                         0  & 0  & -1 & -0.4888 \\
                                                                         0  & -1 & 0  & -0.1333 \\
                                                                         -1 & 0  & 0  & 0.2996  \\
                                                                         0  & 0  & 0  & 1
                                                                     \end{bmatrix}  \begin{bmatrix}
                                                                                        0  & 0  & 1 & 0      \\
                                                                                        -1 & 0  & 0 & 0      \\
                                                                                        0  & -1 & 0 & 0.0997 \\
                                                                                        0  & 0  & 0 & 1
                                                                                    \end{bmatrix}                              \\
                                                                  & =\begin{bmatrix}
                                                                         0 & 1 & 0  & -0.5885 \\
                                                                         1 & 0 & 0  & -0.1333 \\
                                                                         0 & 0 & -1 & 0.2996  \\
                                                                         0 & 0 & 0  & 1
                                                                     \end{bmatrix}                                             \\
        ^{0}T_{6} = \hspace{1pt} ^{0}T_{5} \hspace{1pt} ^{5}T_{6} & =\begin{bmatrix}
                                                                         0 & 1 & 0  & -0.5885 \\
                                                                         1 & 0 & 0  & -0.1333 \\
                                                                         0 & 0 & -1 & 0.2996  \\
                                                                         0 & 0 & 0  & 1
                                                                     \end{bmatrix}  \begin{bmatrix}
                                                                                        -1 & 0  & 0 & 0      \\
                                                                                        0  & -1 & 0 & 0      \\
                                                                                        0  & 0  & 1 & 0.0996 \\
                                                                                        0  & 0  & 0 & 1
                                                                                    \end{bmatrix}                              \\
                                                                  & =\begin{bmatrix}
                                                                         0  & -1 & 0  & -0.5885 \\
                                                                         -1 & 0  & 0  & -0.1333 \\
                                                                         0  & 0  & -1 & 0.2000  \\
                                                                         0  & 0  & 0  & 1
                                                                     \end{bmatrix}
    \end{split}
\end{equation*}

Since these matrices takes the form of,

\begin{equation*}
    ^0T_n(q_1, q_2, ..., q_n) = ^0T_n(\textbf{q}) = \begin{bmatrix}
        ^0R_n(\textbf{q}) & \hspace{1pt}^0o_n(\textbf{q}) \\
        0                 & 1
    \end{bmatrix}
\end{equation*}
\\ And the $i^{th}$ column of the Jacobian matrix for an articulated robot takes the form of,

\begin{equation*}
    J_i = \begin{bmatrix}
        ^0\textbf{z}_{i-1} \times (^0\textbf{o}_{n} - ^0\textbf{o}_{i-1}) \\
        ^0\textbf{z}_{i-1}
    \end{bmatrix}
\end{equation*}

The resulting Jacobian matrix will be:

\begin{equation*}
    \begin{bmatrix}
        0.1333  & -0.0375 & 0.3458  & 0.0996  & 0       & 0  \\
        -0.5885 & 0       & 0       & 0       & -0.0996 & 0  \\
        0       & -0.5885 & -0.4050 & -0.0997 & 0       & 0  \\
        0       & 0       & 0       & 0       & -1      & 0  \\
        0       & -1      & -1      & -1      & 0       & 0  \\
        1       & 0       & 0       & 0       & 0       & -1
    \end{bmatrix}
\end{equation*}