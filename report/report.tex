\documentclass[12pt, a4paper]{article}
\usepackage{graphicx}
\usepackage{mathrsfs,relsize,array}
\usepackage{amsmath}
\usepackage{wrapfig}
\usepackage{float}
\usepackage{amsmath}
\usepackage{array} 
\newcommand\Laplace{\mathlarger{\mathlarger{\mathscr{L}}}}
\title{MTRN4230 - Project 1}
\author{Weichen Tie (z5308889)}
\date{T2 July 2024}

\graphicspath{{./images}}
\begin{document}

\maketitle
\tableofcontents
\section{Part A: Dynamic forward kinematics}
You do not need to include anything in your report for this practical part of the assessment.

\section{Part B: UR5e modelling}
\subsection{Manual Calculation of Forward Kinematic Solutions}
\subsubsection{Resultant Homography Matrix}
\subsubsection{Full Written Working}
\subsubsection{Intermediate Matrices}
\subsubsection{Explaination of the Meaning of Calculated Matrices}
\subsection{Model of UR5e Robotic Arm using RVC Toolbox}
\subsubsection{Forward Kinematic Conversion to Attain Pose with Angles in RPY}
\subsubsection{Matrix Results and Converted Results}
\subsection{Validation of Calculations}
\subsubsection{Screenshot Showing Pose Including the Rotation in RPY Representation}

\section{Part C: Robot Speed Limits}
\subsection{Approach to Calculation}
\subsection{Jacobian Calculation}

\section{Part D: Robot Singularities}
\subsection{Determine the DH matrix}
\subsection{Calculate the Jacobian}
\subsection{For what value(s) is the manipulator at a singularity?}
\subsection{What motion is restricted at this singularity?}
\subsection{What typeof singularity is experienced?}
\end{document}