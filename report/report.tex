\documentclass[12pt]{article}
\usepackage{geometry}
\usepackage{graphicx}
\usepackage{mathrsfs,relsize,array}
\usepackage{amsmath}
\usepackage{wrapfig}
\usepackage{float}
\usepackage{amsmath}
\usepackage{array}
\usepackage{booktabs}
\usepackage{float}
\usepackage{listings}

\geometry{a4paper, vmargin=3.5cm,hmargin={2cm}}

\newcommand\Laplace{\mathlarger{\mathlarger{\mathscr{L}}}}
\title{MTRN4230 - Project 1}
\author{Weichen Tie (z5308889)}
\date{T2 July 2024}

\graphicspath{{./images}}
\begin{document}

\maketitle
\tableofcontents
\section{Part A: Dynamic forward kinematics}
You do not need to include anything in your report for this practical part of the assessment.

\section{Part B: UR5e modelling}

The DH table for the UR5e robot arm is as provided:
\begin{table}[H]
    \centering
    \begin{tabular}{|c|c|c|c|c|}
        \hline
                & \textbf{theta (rad)} & \textbf{a (m)} & \textbf{d (m)} & \textbf{alpha (rad)} \\ \hline
        Joint 1 & 0                    & 0              & 0.1625         & $\pi$/2              \\ \hline
        Joint 2 & 0                    & -0.425         & 0              & 0                    \\ \hline
        Joint 3 & 0                    & -0.3922        & 0              & 0                    \\ \hline
        Joint 4 & 0                    & 0              & 0.1333         & $\pi$/2              \\ \hline
        Joint 5 & 0                    & 0              & 0.0997         & -$\pi$/2             \\ \hline
        Joint 6 & 0                    & 0              & 0.0996         & 0                    \\ \hline
    \end{tabular}
    \caption{The DH table for the UR5e robot arm}
    \label{table:DH-UR5e default}
\end{table}

\textit{The home joint configuration (in  millimeters and degrees)}: [0.00, 77.00, 90.00, -105.00, -90.00, 0.00]

\subsection{Manual Calculation of Forward Kinematic Solutions}
\subsubsection{Resultant Matrix and Output Pose}

The resultant matrix derived for the home joint configuration is (in millimeters and radians):

\begin{equation*}
    ^{0}T_{6} = \begin{bmatrix}
        0      & -0.8829 & 0.4695 & 421.3330  \\
        1.0000 & 0       & 0      & -133.3000 \\
        0      & 0.4695  & 0.8829 & -298.6978 \\
        0      & 0       & 0      & 1.0000
    \end{bmatrix}
\end{equation*}

The resultant posed derived for the home joint configuration is (in millimeters and radians):
\begin{equation*}
    \begin{bmatrix}
        421.3330 & -133.3000 & -298.6978 & 0.4887 & 0 & 1.5708
    \end{bmatrix}
\end{equation*}
\subsubsection{Full Written Working}

From Table \ref{table:DH-UR5e default}, we can derive the following DH table for the home joint configuration:

\begin{table}[H]
    \centering
    \begin{tabular}{|c|c|c|c|c|}
        \hline
                & \textbf{theta (rad)} & \textbf{a (m)} & \textbf{d (m)} & \textbf{alpha (rad)} \\ \hline
        Joint 1 & 0                    & 0              & 0.1625         & 1.5708               \\ \hline
        Joint 2 & 1.3439               & -0.425         & 0              & 0                    \\ \hline
        Joint 3 & 1.5708               & -0.3922        & 0              & 0                    \\ \hline
        Joint 4 & -1.8326              & 0              & 0.1333         & 1.5708               \\ \hline
        Joint 5 & -1.5708              & 0              & 0.0997         & -1.5708              \\ \hline
        Joint 6 & 0                    & 0              & 0.0996         & 0                    \\ \hline
    \end{tabular}
    \caption{The DH table for the UR5e robot arm at home joint configuration}
    \label{table:DH-UR5e home}
\end{table}


From first principles, the homogenous transformation matrix ($^{n-1}T_{n}$) can be derived as follows:

\begin{equation}
    \begin{split}
        ^{n-1}T_{n} & = Rot_{z, \theta_n} Trans_{z, d_n} Trans_{x, a_n} Rot_{x, \alpha_n}                                   \\
                    & = \begin{bmatrix}
                            \cos(\theta_n) & -\sin(\theta_n) & 0 & 0 \\
                            \sin(\theta_n) & \cos(\theta_n)  & 0 & 0 \\
                            0              & 0               & 1 & 0 \\
                            0              & 0               & 0 & 1
                        \end{bmatrix}
        \begin{bmatrix}
            1 & 0 & 0 & 0   \\
            0 & 1 & 0 & 0   \\
            0 & 0 & 1 & d_n \\
            0 & 0 & 0 & 1
        \end{bmatrix}
        \begin{bmatrix}
            1 & 0 & 0 & a_n \\
            0 & 1 & 0 & 0   \\
            0 & 0 & 1 & 0   \\
            0 & 0 & 0 & 1
        \end{bmatrix}
        \begin{bmatrix}
            0 & 0              & 0               & 0 \\
            0 & \cos(\alpha_n) & -\sin(\alpha_n) & 0 \\
            0 & \sin(\alpha_n) & \cos(\alpha_n)  & 0 \\
            0 & 0              & 0               & 1
        \end{bmatrix}                                                                            \\
                    & = \begin{bmatrix}
                            \cos(\theta_n) & -\sin(\theta_n)\cos(\alpha_n) & \sin(\theta_n)\sin(\alpha_n)  & a_n \cos(\theta_n) \\
                            \sin(\theta_n) & \cos(\theta_n)\cos(\alpha_n)  & -\cos(\theta_n)\sin(\alpha_n) & a_n \sin(\theta_n) \\
                            0              & \sin(\alpha_n)                & \cos(\alpha_n)                & d_n                \\
                            0              & 0                             & 0                             & 1
                        \end{bmatrix}
    \end{split}
    \label{equation:DH Matrix}
\end{equation}

Substituting each parameter with its corresponding joint values in the DH table in Table \ref{table:DH-UR5e home} to the transformation matrix in Equation (\ref{equation:DH Matrix}) will yield us the following matrices:

\begin{equation*}
    \begin{split}
        ^{0}T_{1} & = \begin{bmatrix}
                          1.0000 & 0      & 0       & 0        \\
                          0      & 0      & -1.0000 & 0        \\
                          0      & 1.0000 & 0       & 162.5000 \\
                          0      & 0      & 0       & 1.0000
                      \end{bmatrix}  \\
        ^{1}T_{2} & = \begin{bmatrix}
                          0.2250 & -0.9744 & 0      & -95.6042  \\
                          0.9744 & 0.2250  & 0      & -414.1073 \\
                          0      & 0       & 1.0000 & 0         \\
                          0      & 0       & 0      & 1.0000
                      \end{bmatrix} \\
        ^{2}T_{3} & = \begin{bmatrix}
                          0      & -1.0000 & 0      & 0         \\
                          1.0000 & 0       & 0      & -392.2000 \\
                          0      & 0       & 1.0000 & 0         \\
                          0      & 0       & 0      & 1.0000
                      \end{bmatrix} \\
        ^{3}T_{4} & = \begin{bmatrix}
                          -0.2588 & 0      & -0.9659 & 0        \\
                          -0.9659 & 0      & 0.2588  & 0        \\
                          0       & 1.0000 & 0       & 133.3000 \\
                          0       & 0      & 0       & 1.0000
                      \end{bmatrix} \\
        ^{4}T_{5} & = \begin{bmatrix}
                          0       & 0       & 1.0000 & 0       \\
                          -1.0000 & 0       & 0      & 0       \\
                          0       & -1.0000 & 0      & 99.7000 \\
                          0       & 0       & 0      & 1.0000
                      \end{bmatrix}  \\
        ^{5}T_{6} & = \begin{bmatrix}
                          1.0000 & 0      & 0      & 0       \\
                          0      & 1.0000 & 0      & 0       \\
                          0      & 0      & 1.0000 & 99.6000 \\
                          0      & 0      & 0      & 1.0000
                      \end{bmatrix}
    \end{split}
\end{equation*}

And since coordinate frames can be compounded through the relationship $^{A}T_{C} =\hspace{1pt} ^{A}T_{B} \hspace{1pt} ^{B}T_{C}$, we can derive the resultant homogenous matrix $^{0}T_{6}$,
\begin{equation*}
    \begin{split}
        ^{0}T_{6} & = \hspace{1pt} ^{0}T_{1} \hspace{1pt} ^{1}T_{2}\hspace{1pt} ^{2}T_{3}\hspace{1pt} ^{3}T_{4}\hspace{1pt} ^{4}T_{5}\hspace{1pt} ^{5}T_{6} \\
        ^{0}T_{6} & = \begin{bmatrix}
                          0      & -0.8829 & 0.4695 & 421.3330  \\
                          1.0000 & 0       & 0      & -133.3000 \\
                          0      & 0.4695  & 0.8829 & -298.6978 \\
                          0      & 0       & 0      & 1.0000
                      \end{bmatrix}
    \end{split}
\end{equation*}

To get the pose, we realise that the resultant homogenous matrix takes the form of:
\begin{equation*}
    \begin{bmatrix}
        {\begin{array}{ccc|c}&&&\\&R&&T\\&&&\\\hline 0&0&0&1\end{array}}
    \end{bmatrix}
\end{equation*}
So our final joint positions in millimeters are,
\begin{equation*}
    [421.3330, -133.3000, -298.6978]
\end{equation*}
And our roll, pitch and yaw values in radians respectively are can be calulated using Matlab's tr2rpy function,
\begin{center}
    \begin{lstlisting}
    rpy = tr2rpy(R);
    % rpy = [0.4887, 0, 1.5708]
    \end{lstlisting}
\end{center}
And thus, our final pose will be,
\begin{equation*}
    [421.3330, -133.3000, -298.6978, 0.4887, 0, 1.5708]
\end{equation*}
\subsubsection{Intermediate Matrices}
\begin{equation*}
    \begin{split}
        ^{0}T_{1} & = \begin{bmatrix}
                          1.0000 & 0      & 0       & 0        \\
                          0      & 0      & -1.0000 & 0        \\
                          0      & 1.0000 & 0       & 162.5000 \\
                          0      & 0      & 0       & 1.0000
                      \end{bmatrix}    \\
        ^{0}T_{2} & = \begin{bmatrix}
                          0.2250 & -0.9744 & 0       & -95.6042  \\
                          0      & 0       & -1.0000 & 0         \\
                          0.9744 & 0.2250  & 0       & -251.6073 \\
                          0      & 0       & 0       & 1.0000
                      \end{bmatrix}  \\
        ^{0}T_{3} & = \begin{bmatrix}
                          -0.9744 & -0.2250 & 0       & 286.5437  \\
                          0       & 0       & -1.0000 & 0         \\
                          0.2250  & -0.9744 & 0       & -339.8331 \\
                          0       & 0       & 0       & 1.0000
                      \end{bmatrix} \\
        ^{0}T_{4} & = \begin{bmatrix}
                          0.4695 & 0       & 0.8829  & 286.5437  \\
                          0      & -1.0000 & 0       & -133.3000 \\
                          0.8829 & 0       & -0.4695 & -339.8331 \\
                          0      & 0       & 0       & 1.0000
                      \end{bmatrix}  \\
        ^{0}T_{5} & = \begin{bmatrix}
                          0      & -0.8829 & 0.4695 & 374.5736  \\
                          1.0000 & 0       & 0      & -133.3000 \\
                          0      & 0.4695  & 0.8829 & -386.6394 \\
                          0      & 0       & 0      & 1.0000
                      \end{bmatrix}   \\
        ^{0}T_{6} & = \begin{bmatrix}
                          0      & -0.8829 & 0.4695 & 421.3330  \\
                          1.0000 & 0       & 0      & -133.3000 \\
                          0      & 0.4695  & 0.8829 & -298.6978 \\
                          0      & 0       & 0      & 1.0000
                      \end{bmatrix}
    \end{split}
\end{equation*}
\subsubsection{Explaination of the Meaning of Calculated Matrices}



\subsection{Model of UR5e Robotic Arm using RVC Toolbox}
\subsubsection{Forward Kinematic Conversion to Attain Pose with Angles in RPY}
\subsubsection{Matrix Results and Converted Results}
\subsection{Validation of Calculations}
\subsubsection{Screenshot Showing Pose Including the Rotation in RPY Representation}

\section{Part C: Robot Speed Limits}
\subsection{Approach to Calculation}
\subsection{Jacobian Calculation}


\section{Part D: Robot Singularities}
\subsection{Determine the DH matrix}
\begin{table}[H]
    \centering
    \begin{tabular}{|c|c|c|c|c|}
        \hline
                & \textbf{theta (rad)} & \textbf{a (m)} & \textbf{d (m)} & \textbf{alpha (rad)} \\ \hline
        Joint 1 & $\theta_{1}$         & 1              & 0              & 0                    \\ \hline
        Joint 2 & $\theta_{2}$         & 1              & 0              & 0                    \\ \hline
    \end{tabular}
    \caption{The DH table for the 3-Link Robot}
    \label{table:3-Link Robot}
\end{table}
\subsection{Calculate the Jacobian}

\subsection{For what value(s) is the manipulator at a singularity?}
\subsection{What motion is restricted at this singularity?}
\subsection{What type of singularity is experienced?}
\end{document}