\section{Part D: Robot Singularities}
\subsection{Determine the DH matrix}
\begin{table}[H]
    \centering
    \begin{tabular}{|c|c|c|c|c|}
        \hline
                & \textbf{theta (rad)} & \textbf{a (m)} & \textbf{d (m)} & \textbf{alpha (rad)} \\ \hline
        Joint 1 & $q_{1}$              & 1              & 0              & 0                    \\ \hline
        Joint 2 & $q_{2}$              & 1              & 0              & 0                    \\ \hline
        Joint 3 & $q_{3}$              & 1              & 0              & 0                    \\ \hline
    \end{tabular}
    \caption{The DH table for the 3-Link Robot}
    \label{table:3-Link Robot}
\end{table}
\subsection{Calculate the Jacobian}

To calculate the Jacobian that relates the joint velocities to linear velocities of the manipulator,
\\
The forward kinematics solution is as follows,

\begin{equation*}
    T_E = \begin{pmatrix}
        \cos(q_1 + q_2 + q_3) & -\sin(q_1 + q_2 + q_3) & \cos(q_1 + q_2 + q_3) + \cos(q_1 + q_2) + \cos(q_1) \\
        \sin(q_1 + q_2 + q_3) & \cos(q_1 + q_2 + q_3)  & \sin(q_1 + q_2 + q_3) + \sin(q_1 + q_2) + \sin(q_1) \\
        0                     & 0                      & 1                                                   \\
    \end{pmatrix}
\end{equation*}
\\
And since the forward kinematics solution for a 2D transformation matrix takes the form of,
\begin{equation*}
    \begin{bmatrix}
        {\begin{array}{cc|c}R&&T\\&&\\\hline 0&0&1\end{array}}
    \end{bmatrix}
\end{equation*}
We can derive that,
\begin{equation*}
    \textbf{T} = \begin{pmatrix}
        x \\ y
    \end{pmatrix} = \begin{pmatrix}
        \cos(q_1 + q_2 + q_3) + \cos(q_1 + q_2) + \cos(q_1) \\
        \sin(q_1 + q_2 + q_3) + \sin(q_1 + q_2) + \sin(q_1)
    \end{pmatrix} \\
\end{equation*}
then,
\begin{equation*}
    \begin{split}
        \frac{\delta x}{\delta q_1} & = -\sin(q_1 + q_2 + q_3) - \sin(q_1 + q_2) - \sin(q_1) \\
        \frac{\delta x}{\delta q_2} & = -\sin(q_1 + q_2 + q_3) - \sin(q_1 + q_2)             \\
        \frac{\delta x}{\delta q_3} & = -\sin(q_1 + q_2 + q_3)                               \\
        \frac{\delta y}{\delta q_1} & =  \cos(q_1 + q_2 + q_3) + \cos(q_1 + q_2) + \cos(q_1) \\
        \frac{\delta y}{\delta q_2} & =  \cos(q_1 + q_2 + q_3) + \cos(q_1 + q_2)             \\
        \frac{\delta y}{\delta q_3} & = \cos(q_1 + q_2 + q_3)
    \end{split}
\end{equation*}
So the Jacobian matrix (\textbf{J$_v$}) that relates joint velocity $ \begin{pmatrix}
        \dot{q_1} \\ \dot{q_2} \\ \dot{q_3}
    \end{pmatrix} $ to spacial velocities $ \begin{pmatrix}
        \dot{x} \\ \dot{y}
    \end{pmatrix} $ takes the form of,
\begin{equation*}
    \begin{pmatrix}
        \dot{x} \\ \dot{y}
    \end{pmatrix} = \textbf{J$_v$} \begin{pmatrix}
        \dot{q_1} \\ \dot{q_2} \\ \dot{q_3}
    \end{pmatrix}
\end{equation*}
where,
\begin{equation*}
    \begin{split}
         & \textbf{J$_v$}  = \begin{pmatrix}
                                 \frac{\delta x}{\delta q_1} & \frac{\delta x}{\delta q_2} & \frac{\delta x}{\delta q_3} \\
                                 \frac{\delta y}{\delta q_1} & \frac{\delta y}{\delta q_2} & \frac{\delta y}{\delta q_3}
                             \end{pmatrix}                  \\
         & = \begin{pmatrix}
                 -\sin(q_1 + q_2 + q_3) - \sin(q_1 + q_2) - \sin(q_1) & -\sin(q_1 + q_2 + q_3) - \sin(q_1 + q_2) & -\sin(q_1 + q_2 + q_3) \\
                 \cos(q_1 + q_2 + q_3) + \cos(q_1 + q_2) + \cos(q_1)  & \cos(q_1 + q_2 + q_3) + \cos(q_1 + q_2)  & \cos(q_1 + q_2 + q_3)
             \end{pmatrix}
    \end{split}
\end{equation*}
\\
The Jacobian that relates the joint velocities with angular velocities can be derived as follows,
\begin{equation*}
    \begin{split}
        \textbf{R} & = \begin{pmatrix}
                           \cos(q_1 + q_2 + q_3) & -\sin(q_1 + q_2 + q_3) \\
                           \sin(q_1 + q_2 + q_3) & \cos(q_1 + q_2 + q_3)
                       \end{pmatrix} \\
                   & = \begin{pmatrix}
                           \cos(\theta) & -\sin(\theta) \\
                           \sin(\theta) & \cos(\theta)
                       \end{pmatrix}
    \end{split}
\end{equation*}
\\ where, by inspection,
\begin{equation*}
    \begin{split}
        \theta                         & = q_1 + q_2 + q_3                                                                                           \\
        \therefore \hspace{14pt}\omega & = \dot{\theta} = \dot{q_1} + \dot{q_2} + \dot{q_3}                                                          \\
        \therefore \hspace{14pt}\omega & = \begin{pmatrix} 1 & 1 & 1 \end{pmatrix} \begin{pmatrix} \dot{q_1} \\ \dot{q_2} \\ \dot{q_3} \end{pmatrix}
    \end{split}
\end{equation*}
\\ and hence,
\begin{equation*}
    \textbf{J$_\omega$} = \begin{pmatrix} 1 & 1 & 1 \end{pmatrix}
\end{equation*}
Therefore the full Jacobian matrix is as follows,
\begin{equation*}
    \textbf{J} = \begin{pmatrix}
        -\sin(q_1 + q_2 + q_3) - \sin(q_1 + q_2) - \sin(q_1) & -\sin(q_1 + q_2 + q_3) - \sin(q_1 + q_2) & -\sin(q_1 + q_2 + q_3) \\
        \cos(q_1 + q_2 + q_3) + \cos(q_1 + q_2) + \cos(q_1)  & \cos(q_1 + q_2 + q_3) + \cos(q_1 + q_2)  & \cos(q_1 + q_2 + q_3)  \\
        1                                                    & 1                                        & 1
    \end{pmatrix}
\end{equation*}
\subsection{For what value(s) is the manipulator at a singularity?}
To calculate the values of the manipulator at a singularity, we check the values for which the Jacobian determinant equals 0.

\begin{equation*}
    \det(J) = 0
\end{equation*}
\\

\subsection{What motion is restricted at this singularity?}

\subsection{What type of singularity is experienced?}