\section{Part D: Robot Singularities}
\subsection{Determine the DH matrix}
\begin{table}[H]
    \centering
    \begin{tabular}{|c|c|c|c|c|}
        \hline
                & \textbf{theta (rad)} & \textbf{a (m)} & \textbf{d (m)} & \textbf{alpha (rad)} \\ \hline
        Joint 1 & $q_{1}$              & 1              & 0              & 0                    \\ \hline
        Joint 2 & $q_{2}$              & 1              & 0              & 0                    \\ \hline
        Joint 3 & $q_{3}$              & 1              & 0              & 0                    \\ \hline
    \end{tabular}
    \caption{The DH table for the 3-Link Robot}
    \label{table:3-Link Robot}
\end{table}
\subsection{Calculate the Jacobian}

To calculate the Jacobian that relates the joint velocities to linear velocities of the manipulator, we need to first derive the homogenous transfromation matrix $^{0}T_{3}$ where,
\begin{equation*}
    ^{0}T_{3}    = \hspace{1pt} ^{0}T_{1} \hspace{1pt} ^{1}T_{2}\hspace{1pt} ^{2}T_{3}\hspace{1pt}\\
\end{equation*}
\\and,
\begin{equation*}
    ^{n-1}T_{n}  = \begin{bmatrix}
        \cos(q_n) & -\sin(q_n)\cos(\alpha_n) & \sin(q_n)\sin(\alpha_n)  & a_n \cos(q_n) \\
        \sin(q_n) & \cos(q_n)\cos(\alpha_n)  & -\cos(q_n)\sin(\alpha_n) & a_n \sin(q_n) \\
        0         & \sin(\alpha_n)           & \cos(\alpha_n)           & d_n           \\
        0         & 0                        & 0                        & 1
    \end{bmatrix}
\end{equation*}
\\By substituting the DH parameters in table \ref{table:3-Link Robot} into $^{n-1}T_{n}$ we get,
\begin{equation*}
    ^{n-1}T_{n}  = \begin{bmatrix}
        \cos(q_n) & -\sin(q_n) & 0 & \cos(q_n) \\
        \sin(q_n) & \cos(q_n)  & 0 & \sin(q_n) \\
        0         & 0          & 1 & 0         \\
        0         & 0          & 0 & 1
    \end{bmatrix}
\end{equation*}
\\which can be simplified to a 2D transformation matrix by removing the third row and column,
\begin{equation*}
    ^{n-1}T_{n}  = \begin{bmatrix}
        \cos(q_n) & -\sin(q_n) & \cos(q_n) \\
        \sin(q_n) & \cos(q_n)  & \sin(q_n) \\
        0         & 0          & 1
    \end{bmatrix}
\end{equation*}
\\ We can now derive the full 2D homogenous transformation matrix $^0T_3$ by chaining transformation matrices.
\begin{equation*}
    \begin{split}
        ^{0}T_{3} & = \hspace{1pt} ^{0}T_{1} \hspace{1pt} ^{1}T_{2}\hspace{1pt} ^{2}T_{3}                                  \\
                  & = \begin{bmatrix}
                          \cos(q_1) & -\sin(q_1) & \cos(q_1) \\
                          \sin(q_1) & \cos(q_1)  & \sin(q_1) \\
                          0         & 0          & 1
                      \end{bmatrix}
        \begin{bmatrix}
            \cos(q_2) & -\sin(q_2) & \cos(q_2) \\
            \sin(q_2) & \cos(q_2)  & \sin(q_2) \\
            0         & 0          & 1
        \end{bmatrix}
        \begin{bmatrix}
            \cos(q_3) & -\sin(q_3) & \cos(q_3) \\
            \sin(q_3) & \cos(q_3)  & \sin(q_3) \\
            0         & 0          & 1
        \end{bmatrix}                                                                                 \\
                  & = \begin{bmatrix}
                          \cos(q_1 + q_2 + q_3) & -\sin(q_1 + q_2 + q_3) & \cos(q_1 + q_2 + q_3) + \cos(q_1 + q_2) + \cos(q_1) \\
                          \sin(q_1 + q_2 + q_3) & \cos(q_1 + q_2 + q_3)  & \sin(q_1 + q_2 + q_3) + \sin(q_1 + q_2) + \sin(q_1) \\
                          0                     & 0                      & 1                                                   \\
                      \end{bmatrix}
    \end{split}
\end{equation*}
\\ And since the forward kinematics solution for a 2D transformation matrix takes the form of,
\begin{equation*}
    \begin{bmatrix}
        {\begin{array}{cc|c}R&&T\\&&\\\hline 0&0&1\end{array}}
    \end{bmatrix}
\end{equation*}
We can derive that,
\begin{equation*}
    \textbf{T} = \begin{bmatrix}
        x \\ y
    \end{bmatrix} = \begin{bmatrix}
        \cos(q_1 + q_2 + q_3) + \cos(q_1 + q_2) + \cos(q_1) \\
        \sin(q_1 + q_2 + q_3) + \sin(q_1 + q_2) + \sin(q_1)
    \end{bmatrix} \\
\end{equation*}
then,
\begin{equation*}
    \begin{split}
        \frac{\delta x}{\delta q_1} & = -\sin(q_1 + q_2 + q_3) - \sin(q_1 + q_2) - \sin(q_1) \\
        \frac{\delta x}{\delta q_2} & = -\sin(q_1 + q_2 + q_3) - \sin(q_1 + q_2)             \\
        \frac{\delta x}{\delta q_3} & = -\sin(q_1 + q_2 + q_3)                               \\
        \frac{\delta y}{\delta q_1} & =  \cos(q_1 + q_2 + q_3) + \cos(q_1 + q_2) + \cos(q_1) \\
        \frac{\delta y}{\delta q_2} & =  \cos(q_1 + q_2 + q_3) + \cos(q_1 + q_2)             \\
        \frac{\delta y}{\delta q_3} & = \cos(q_1 + q_2 + q_3)
    \end{split}
\end{equation*}
So the Jacobian matrix (\textbf{J$_v$}) that relates joint velocity $ \begin{bmatrix}
        \dot{q_1} \\ \dot{q_2} \\ \dot{q_3}
    \end{bmatrix} $ to spacial velocities $ \begin{bmatrix}
        \dot{x} \\ \dot{y}
    \end{bmatrix} $ takes the form of,
\begin{equation*}
    \begin{bmatrix}
        \dot{x} \\ \dot{y}
    \end{bmatrix} = \textbf{J$_v$} \begin{bmatrix}
        \dot{q_1} \\ \dot{q_2} \\ \dot{q_3}
    \end{bmatrix}
\end{equation*}
where,
\begin{equation*}
    \begin{split}
         & \textbf{J$_v$}  = \begin{bmatrix}
                                 \frac{\delta x}{\delta q_1} & \frac{\delta x}{\delta q_2} & \frac{\delta x}{\delta q_3} \\
                                 \frac{\delta y}{\delta q_1} & \frac{\delta y}{\delta q_2} & \frac{\delta y}{\delta q_3}
                             \end{bmatrix}                  \\
         & = \begin{bmatrix}
                 -\sin(q_1 + q_2 + q_3) - \sin(q_1 + q_2) - \sin(q_1) & -\sin(q_1 + q_2 + q_3) - \sin(q_1 + q_2) & -\sin(q_1 + q_2 + q_3) \\
                 \cos(q_1 + q_2 + q_3) + \cos(q_1 + q_2) + \cos(q_1)  & \cos(q_1 + q_2 + q_3) + \cos(q_1 + q_2)  & \cos(q_1 + q_2 + q_3)
             \end{bmatrix}
    \end{split}
\end{equation*}
\\
The Jacobian that relates the joint velocities with angular velocities can be derived as follows,
\begin{equation*}
    \textbf{R} = \begin{bmatrix}
        \cos(\theta) & -\sin(\theta) \\
        \sin(\theta) & \cos(\theta)
    \end{bmatrix} = \begin{bmatrix}
        \cos(q_1 + q_2 + q_3) & -\sin(q_1 + q_2 + q_3) \\
        \sin(q_1 + q_2 + q_3) & \cos(q_1 + q_2 + q_3)
    \end{bmatrix}
\end{equation*}
\\ where, by inspection,
\begin{equation*}
    \begin{split}
        \theta                         & = q_1 + q_2 + q_3                                                                                           \\
        \therefore \hspace{14pt}\omega & = \dot{\theta} = \dot{q_1} + \dot{q_2} + \dot{q_3}                                                          \\
        \therefore \hspace{14pt}\omega & = \begin{bmatrix} 1 & 1 & 1 \end{bmatrix} \begin{bmatrix} \dot{q_1} \\ \dot{q_2} \\ \dot{q_3} \end{bmatrix}
    \end{split}
\end{equation*}
\\ and hence,
\begin{equation*}
    \textbf{J$_\omega$} = \begin{bmatrix} 1 & 1 & 1 \end{bmatrix}
\end{equation*}
Therefore the full Jacobian matrix is as follows,
\begin{equation*}
    \textbf{J} = \begin{bmatrix}
        -\sin(q_1 + q_2 + q_3) - \sin(q_1 + q_2) - \sin(q_1) & -\sin(q_1 + q_2 + q_3) - \sin(q_1 + q_2) & -\sin(q_1 + q_2 + q_3) \\
        \cos(q_1 + q_2 + q_3) + \cos(q_1 + q_2) + \cos(q_1)  & \cos(q_1 + q_2 + q_3) + \cos(q_1 + q_2)  & \cos(q_1 + q_2 + q_3)  \\
        1                                                    & 1                                        & 1
    \end{bmatrix}
\end{equation*}
\subsection{For what value(s) is the manipulator at a singularity?}
To calculate the values of the manipulator at a singularity, we check the values for which the Jacobian determinant equals 0.

\begin{equation*}
    \det(J) = 0
\end{equation*}
\\ To simplify the calculation process, let,
\begin{equation*}
    \begin{split}
        a & = \sin(q_1 + q_2 + q_3) \\
        b & = \sin(q_1 + q_2)       \\
        c & = \sin(q_1)             \\
        d & = \cos(q_1 + q_2 + q_3) \\
        e & = \cos(q_1 + q_2)       \\
        f & = \cos(q_1)             \\
    \end{split}
\end{equation*}
\\ So that our Jacobian matrix is,
\begin{equation*}
    \textbf{J} = \begin{bmatrix}
        -a - b - c & -a - b & -a \\
        d + e + f  & d + e  & d  \\
        1          & 1      & 1
    \end{bmatrix}
\end{equation*}
And as such, solving for singularity positions through the Jacobian determinant $\det(\textbf{J}) = 0$ equates to,
\begin{equation*}
    \begin{split}
        \det(\textbf{J}) = 0 & = (-a - b - c)[(d + e) - (d)] - (-a - b)[(d + e + f) - (d)] + (-a)[(d+e+f)-(d+e)]                            \\
                             & = -e(a + b + c) + (a+b)(e + f) - af                                                                          \\
                             & = -ae -be - ce + ae + af + be + bf -af                                                                       \\
                             & = bf - ce                                                                                                    \\
                             & = \sin(q_1 + q_2)\cos(q_1)  - \sin(q_1)\cos(q_1 + q_2)                                                       \\
                             & = \sin(q_2) \hspace{170pt}\because \sin(\alpha - \beta) = \sin(\alpha)\cos(\beta)  - \sin(\beta)\cos(\alpha)
    \end{split}
\end{equation*}
Therefore the singularities exist whenever $\sin(q_2) = 0$ or in other words,
\begin{equation*}
    q_2 = n\pi, \hspace{12pt}n\in\mathbb{Z}
\end{equation*}
This occurs whenever link 2 is colinear with link 1.
\subsection{What motion is restricted at this singularity?}

\subsection{What type of singularity is experienced?}