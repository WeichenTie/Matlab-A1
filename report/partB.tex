\section{Part B: UR5e modelling}

The DH table for the UR5e robot arm is as provided:
\begin{table}[H]
    \centering
    \begin{tabular}{|c|c|c|c|c|}
        \hline
                & \textbf{theta (rad)} & \textbf{a (m)} & \textbf{d (m)} & \textbf{alpha (rad)} \\ \hline
        Joint 1 & 0                    & 0              & 0.1625         & $\pi$/2              \\ \hline
        Joint 2 & 0                    & -0.425         & 0              & 0                    \\ \hline
        Joint 3 & 0                    & -0.3922        & 0              & 0                    \\ \hline
        Joint 4 & 0                    & 0              & 0.1333         & $\pi$/2              \\ \hline
        Joint 5 & 0                    & 0              & 0.0997         & -$\pi$/2             \\ \hline
        Joint 6 & 0                    & 0              & 0.0996         & 0                    \\ \hline
    \end{tabular}
    \caption{The DH table for the UR5e robot arm}
    \label{table:DH-UR5e default}
\end{table}

\textit{The home joint configuration (in degrees)}: [0.00, -75.00, 90.00, -105.00, -90.00, 0]

\subsection{Manual Calculation of Forward Kinematic Solutions}
\subsubsection{Resultant Matrix and Output Pose}
The resultant matrix derived for the home joint configuration is (in millimeters and radians):

\begin{equation*}
    ^{0}T_{6} = \begin{bmatrix}
        0    & 1.00 & 0     & -588.53 \\
        1.00 & 0    & 0     & -133.30 \\
        0    & 0    & -1.00 & 371.91  \\
        0    & 0    & 0     & 1.00
    \end{bmatrix}
\end{equation*}
\\The resultant posed derived for the home joint configuration is (in millimeters and radians):
\begin{equation*}
    \begin{bmatrix}
        -588.5342 & -133.3000 & 371.9096 & 3.1416 & 0 & 1.5708
    \end{bmatrix}
\end{equation*}
\\ However upon cross inspection with the simulation, the 5th joint had an angle of -3.1416, but this is equivalent as it is a phase difference of 2 $\pi$.
\subsubsection{Full Written Working}

From Table \ref{table:DH-UR5e default}, we can derive the following DH table for the home joint configuration:

\begin{table}[H]
    \centering
    \begin{tabular}{|c|c|c|c|c|}
        \hline
                & \textbf{theta (rad)} & \textbf{a (m)} & \textbf{d (m)} & \textbf{alpha (rad)} \\ \hline
        Joint 1 & 0                    & 0              & 0.1625         & 1.5708               \\ \hline
        Joint 2 & 1.3439               & -0.425         & 0              & 0                    \\ \hline
        Joint 3 & 1.5708               & -0.3922        & 0              & 0                    \\ \hline
        Joint 4 & -1.8326              & 0              & 0.1333         & 1.5708               \\ \hline
        Joint 5 & -1.5708              & 0              & 0.0997         & -1.5708              \\ \hline
        Joint 6 & 0                    & 0              & 0.0996         & 0                    \\ \hline
    \end{tabular}
    \caption{The DH table for the UR5e robot arm at home joint configuration}
    \label{table:DH-UR5e home}
\end{table}


From first principles, the homogenous transformation matrix ($^{n-1}T_{n}$) can be derived as follows:

\begin{equation}
    \begin{split}
        ^{n-1}T_{n} & = Rot_{z, \theta_n} Trans_{z, d_n} Trans_{x, a_n} Rot_{x, \alpha_n}                                   \\
                    & = \begin{bmatrix}
                            \cos(\theta_n) & -\sin(\theta_n) & 0 & 0 \\
                            \sin(\theta_n) & \cos(\theta_n)  & 0 & 0 \\
                            0              & 0               & 1 & 0 \\
                            0              & 0               & 0 & 1
                        \end{bmatrix}
        \begin{bmatrix}
            1 & 0 & 0 & 0   \\
            0 & 1 & 0 & 0   \\
            0 & 0 & 1 & d_n \\
            0 & 0 & 0 & 1
        \end{bmatrix}
        \begin{bmatrix}
            1 & 0 & 0 & a_n \\
            0 & 1 & 0 & 0   \\
            0 & 0 & 1 & 0   \\
            0 & 0 & 0 & 1
        \end{bmatrix}
        \begin{bmatrix}
            0 & 0              & 0               & 0 \\
            0 & \cos(\alpha_n) & -\sin(\alpha_n) & 0 \\
            0 & \sin(\alpha_n) & \cos(\alpha_n)  & 0 \\
            0 & 0              & 0               & 1
        \end{bmatrix}                                                                            \\
                    & = \begin{bmatrix}
                            \cos(\theta_n) & -\sin(\theta_n)\cos(\alpha_n) & \sin(\theta_n)\sin(\alpha_n)  & a_n \cos(\theta_n) \\
                            \sin(\theta_n) & \cos(\theta_n)\cos(\alpha_n)  & -\cos(\theta_n)\sin(\alpha_n) & a_n \sin(\theta_n) \\
                            0              & \sin(\alpha_n)                & \cos(\alpha_n)                & d_n                \\
                            0              & 0                             & 0                             & 1
                        \end{bmatrix}
    \end{split}
    \label{equation:DH Matrix}
\end{equation}

Substituting each parameter with its corresponding joint values in the DH table in Table \ref{table:DH-UR5e home} to the transformation matrix in Equation (\ref{equation:DH Matrix}) will yield us the following matrices:

\begin{equation*}
    \begin{split}
        ^{0}T_{1} & = \begin{bmatrix}
                          1.00 & 0    & 0     & 0      \\
                          0    & 0    & -1.00 & 0      \\
                          0    & 1.00 & 0     & 162.50 \\
                          0    & 0    & 0     & 1.0000
                      \end{bmatrix}    \\
        ^{1}T_{2} & = \begin{bmatrix}
                          0.26 & -0.97 & 0    & -110.00 \\
                          0.97 & 0.26  & 0    & -410.52 \\
                          0    & 0     & 1.00 & 0       \\
                          0    & 0     & 0    & 1.00
                      \end{bmatrix}   \\
        ^{2}T_{3} & = \begin{bmatrix}
                          0    & -1.00 & 0    & 0       \\
                          1.00 & 0     & 0    & -392.20 \\
                          0    & 0     & 1.00 & 0       \\
                          0    & 0     & 0    & 1.00
                      \end{bmatrix}   \\
        ^{3}T_{4} & = \begin{bmatrix}
                          -0.26 & 0      & -0.97 & 0      \\
                          -0.97 & 0      & 0.26  & 0      \\
                          0     & 1.0000 & 0     & 133.30 \\
                          0     & 0      & 0     & 1.00
                      \end{bmatrix} \\
        ^{4}T_{5} & = \begin{bmatrix}
                          0     & 0     & 1.00 & 0     \\
                          -1.00 & 0     & 0    & 0     \\
                          0     & -1.00 & 0    & 99.70 \\
                          0     & 0     & 0    & 1.00
                      \end{bmatrix}    \\
        ^{5}T_{6} & = \begin{bmatrix}
                          1.00 & 0    & 0    & 0     \\
                          0    & 1.00 & 0    & 0     \\
                          0    & 0    & 1.00 & 99.60 \\
                          0    & 0    & 0    & 1.00
                      \end{bmatrix}
    \end{split}
\end{equation*}

And since coordinate frames can be compounded through the relationship $^{A}T_{C} =\hspace{1pt} ^{A}T_{B} \hspace{1pt} ^{B}T_{C}$, we can derive the resultant homogenous matrix $^{0}T_{6}$,
\begin{equation*}
    \begin{split}
        ^{0}T_{6} & = \hspace{1pt} ^{0}T_{1} \hspace{1pt} ^{1}T_{2}\hspace{1pt} ^{2}T_{3}\hspace{1pt} ^{3}T_{4}\hspace{1pt} ^{4}T_{5}\hspace{1pt} ^{5}T_{6} \\
        ^{0}T_{6} & = \begin{bmatrix}
                          0    & 1.00 & 0     & -588.53 \\
                          1.00 & 0    & 0     & -133.30 \\
                          0    & 0    & -1.00 & 371.91  \\
                          0    & 0    & 0     & 1.00
                      \end{bmatrix}
    \end{split}
\end{equation*}\\
To get the pose, we realise that the resultant homogenous matrix takes the form of:

\begin{equation*}
    \begin{bmatrix}
        {\begin{array}{ccc|c}&&&\\&R&&T\\&&&\\\hline 0&0&0&1\end{array}}
    \end{bmatrix}
\end{equation*}
So our final joint positions in millimeters are,
\begin{equation*}
    [-588.5342, -133.3000, 371.9096]
\end{equation*}
And our roll, pitch and yaw values in radians respectively are can be calculated using Matlab's tr2rpy function,
\begin{center}
    \begin{lstlisting}
    rpy = tr2rpy(R);
    % rpy = [3.1416, 0, 1.5708]
    \end{lstlisting}
\end{center}
And thus, our final pose will be,
\begin{equation*}
    [-588.5342, -133.3000, 371.9096, 3.1416, 0, 1.5708]
\end{equation*}
\subsubsection{Intermediate Matrices}
\begin{equation*}
    \begin{split}
        ^{0}T_{1}                                                 & = \begin{bmatrix}
                                                                          1.00 & 0    & 0     & 0      \\
                                                                          0    & 0    & -1.00 & 0      \\
                                                                          0    & 1.00 & 0     & 162.50 \\
                                                                          0    & 0    & 0     & 1.00
                                                                      \end{bmatrix}    \\
        ^{0}T_{2} = \hspace{1pt} ^{0}T_{1} \hspace{1pt} ^{1}T_{2} & = \begin{bmatrix}
                                                                          0.26  & 0.97 & 0     & -110.00 \\
                                                                          0     & 0    & -1.00 & 0       \\
                                                                          -0.97 & 0.26 & 0     & 573.02  \\
                                                                          0     & 0    & 0     & 1.00
                                                                      \end{bmatrix}  \\
        ^{0}T_{3} = \hspace{1pt} ^{0}T_{2} \hspace{1pt} ^{2}T_{3} & = \begin{bmatrix}
                                                                          0.97 & -0.26 & 0     & -488.83 \\
                                                                          0    & 0     & -1.00 & 0       \\
                                                                          0.26 & 0.97  & 0     & 471.51  \\
                                                                          0    & 0     & 0     & 1.00
                                                                      \end{bmatrix}  \\
        ^{0}T_{4} = \hspace{1pt} ^{0}T_{3} \hspace{1pt} ^{3}T_{4} & = \begin{bmatrix}
                                                                          0     & 0     & -1.00 & -488.83 \\
                                                                          0     & -1.00 & 0     & -133.30 \\
                                                                          -1.00 & 0     & 0     & 471.51  \\
                                                                          0     & 0     & 0     & 1.00
                                                                      \end{bmatrix} \\
        ^{0}T_{5} = \hspace{1pt} ^{0}T_{4} \hspace{1pt} ^{4}T_{5} & = \begin{bmatrix}
                                                                          0    & 1.00 & 0     & -588.53 \\
                                                                          1.00 & 0    & 0     & -133.30 \\
                                                                          0    & 0    & -1.00 & 471.51  \\
                                                                          0    & 0    & 0     & 1.00
                                                                      \end{bmatrix}   \\
        ^{0}T_{6} = \hspace{1pt} ^{0}T_{5} \hspace{1pt} ^{5}T_{6} & = \begin{bmatrix}
                                                                          0    & 1.00 & 0     & -588.53 \\
                                                                          1.00 & 0    & 0     & -133.30 \\
                                                                          0    & 0    & -1.00 & 371.91  \\
                                                                          0    & 0    & 0     & 1.00
                                                                      \end{bmatrix}
    \end{split}
\end{equation*}
\subsubsection{Explanation of the Meaning of Calculated Matrices}



\subsection{Model of UR5e Robotic Arm using RVC Toolbox}
\subsubsection{Forward Kinematic Conversion to Attain Pose with Angles in RPY}
\subsubsection{Matrix Results and Converted Results}
\subsection{Validation of Calculations}
\subsubsection{Screenshot Showing Pose Including the Rotation in RPY Representation}
